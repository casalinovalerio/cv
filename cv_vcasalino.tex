%%%%%%%%%%%%%%%%%
% This is an sample CV template created using altacv.cls
% (v1.1.4, 27 July 2018) written by LianTze Lim (liantze@gmail.com). Now compiles with pdfLaTeX, XeLaTeX and LuaLaTeX.
% 
%% It may be distributed and/or modified under the
%% conditions of the LaTeX Project Public License, either version 1.3
%% of this license or (at your option) any later version.
%% The latest version of this license is in
%%    http://www.latex-project.org/lppl.txt
%% and version 1.3 or later is part of all distributions of LaTeX
%% version 2003/12/01 or later.
%%%%%%%%%%%%%%%%

%% If you need to pass whatever options to xcolor
\PassOptionsToPackage{dvipsnames}{xcolor}

%% If you are using \orcid or academicons
%% icons, make sure you have the academicons 
%% option here, and compile with XeLaTeX
%% or LuaLaTeX.
% \documentclass[10pt,a4paper,academicons]{altacv}

%% Use the "normalphoto" option if you want a normal photo instead of cropped to a circle
% \documentclass[10pt,a4paper,normalphoto]{altacv}

\documentclass[10pt,a4paper]{altacv}
%% AltaCV uses the fontawesome and academicon fonts
%% and packages. 
%% See texdoc.net/pkg/fontawecome and http://texdoc.net/pkg/academicons for full list of symbols.
%% 
%% Compile with LuaLaTeX for best results. If you
%% want to use XeLaTeX, you may need to install
%% Academicons.ttf in your operating system's font 
%% folder.


% Change the page layout if you need to
\geometry{left=1cm,right=9cm,marginparwidth=6.8cm,marginparsep=1.2cm,top=1.25cm,bottom=1.25cm,footskip=2\baselineskip}

% Change the font if you want to.

% If using pdflatex:
\usepackage[T1]{fontenc}
\usepackage[utf8]{inputenc}
\usepackage[default]{lato}

% If using xelatex or lualatex:
% \setmainfont{Lato}

% Change the colours if you want to
\definecolor{Mulberry}{HTML}{72243D}
\definecolor{SlateGrey}{HTML}{2E2E2E}
\definecolor{LightGrey}{HTML}{666666}
\colorlet{heading}{Sepia}
\colorlet{accent}{Mulberry}
\colorlet{emphasis}{SlateGrey}
\colorlet{body}{LightGrey}

% Change the bullets for itemize and rating marker
% for \cvskill if you want to
\renewcommand{\itemmarker}{{\small\textbullet}}
\renewcommand{\ratingmarker}{\faCircle}
%% sample.bib contains your publications
%\addbibresource{sample.bib}

\usepackage{hyperref}

\begin{document}

\name{VALERIO CASALINO}
\tagline{Studente di Ingegneria Informatica PoliTo}
\photo{3cm}{./fototessera.jpg}
\personalinfo{%
  % Not all of these are required!
  % You can add your own with \printinfo{symbol}{detail}
  \email{\href{mailto:casalinovalerio.cv@gmail.com}{casalinovalerio.cv@gmail.com}}
  %\phone{\href{tel:}{}}
  %\mailaddress{Elambilakkal(Hose),Calicut airport(Po),Kondotty, Malappuram }
  %\twitter{@twitterhandle}
  \linkedin{\url{https://www.linkedin.com/in/valerio-casalino/}}
  \github{\url{https://github.com/casalinovalerio}}
  %% You MUST add the academicons option to \documentclass, then compile with LuaLaTeX or XeLaTeX, if you want to use \orcid or other academicons commands.
  %\orcid{orcid.org/0000-0000-0000-0000}
  \homepage{Copia: \url{http://bit.ly/vcasalino-cv-it} }
  \location{Roma, Italia}
}

%% Make the header extend all the way to the right, if you want. 
\begin{fullwidth}
\makecvheader
\end{fullwidth}

%% Depending on your tastes, you may want to make fonts of itemize environments slightly smaller
% \AtBeginEnvironment{itemize}{\small}

%\cvsection{Esami Sostenuti}

% Adapted from @Jake's answer from http://tex.stackexchange.com/a/82729/226
% \wheelchart{outer radius}{inner radius}{
% comma-separated list of value/text width/color/detail}

%\wheelchart{1cm}{0.5cm}{%
%  129/8em/accent!30/{CFU Ottenuti \\ (129)}, 
%  54/8em/accent!8/{CFU Mancanti \\(54)}
%}

\cvsection[page1sidebar]{Esperienze}

%\cvevent{Sport}{}{}{}{Ne ho sempre praticato almeno uno: judo, nuoto, scherma (agonistica), sci, prepugilistica e boxe.}

\cvevent{Student internship}{Spike Reply}{Marzo-Aprile 2019}{Roma, Italia}{Studio e pratica di web application penetration test, vulnerability assessment e common vulnerabilities and exposures and remediation.}

\divider

\cvevent{Consigliere di amministrazione}{ONLUS T.G.S. Ongos 2000}{2015 - 2017}{Roma, Italia}

\divider

\cvevent{Seminario su tecniche di negoziazione}{Club Canova Giovane - LUISS}{Marzo 2016}{Roma, Italia}

% Le conoscenze apprese mi sono servite per svolgere il ruolo di barman per diverse feste con un gran numero di invitati.

\divider

\cvevent{Corso da bartender}{Flair Project}{Novembre 2015}{Roma, Italia}{Durante il quinto anno di liceo, ho frequentato un corso serale di bartending americano. Ho iniziato ad essere invitato a più feste...}

\divider

\cvevent{Scoutismo}{Federazione Scout d'Europa (FSE)}{2007 - 2014}{Roma, Italia}{Sono stato membro e leader della mia squadra, fino a che non mi sono allontanato per dedicarmi di più allo studio.}

%% Provide the file name containing the sidebar contents as an optional parameter to \cvsection.
%% You can always just use \marginpar{...} if you do
%% not need to align the top of the contents to any
%% \cvsection title in the "main" bar.
\cvsection{Esperienze Internazionali}

\cvevent{Orientamento universitario e progetto di ricerca}{STEM Research Experience, University of Colorado}{29/06/2015 - 23/07/2015}{Boulder(CO), USA}

\divider

\cvevent{General English Course}{Dublin City University}{11/08/2014 - 29/08/2014}{Dublino, Irlanda}

\divider

\cvevent{General English Course}{University of Bath}{22/07/2012 - 05/08/2012}{Bath, Regno Unito}



%\cvsection{Projects}

%\cvevent{Project 1}{Funding agency/institution}{Project duration}{}
%\begin{itemize}
%\item Details
%\end{itemize}

%\divider

%\cvevent{Project 2}{Funding agency/institution}{Project duration}{}
%A short abstract would also work.

\bigskip

\begin{fullwidth}
\small
	\begin{center}
		Autorizzo il trattamento dei dati personali (art.13 D. Lgs. 30 Giugno 2003 n.196).
	\end{center}
\end{fullwidth}

%\clearpage
%\cvsection[page2sidebar]{Publications}

%\nocite{*}

%\printbibliography[heading=pubtype,title={\printinfo{\faBook}{Books}},type=book]

%\divider

%\printbibliography[heading=pubtype,title={\printinfo{\faFileTextO}{Journal Articles}},type=article]

%\divider

%\printbibliography[heading=pubtype,title={\printinfo{\faGroup}{Conference Proceedings}},type=inproceedings]

%% If the NEXT page doesn't start with a \cvsection but you'd
%% still like to add a sidebar, then use this command on THIS
%% page to add it. The optional argument lets you pull up the 
%% sidebar a bit so that it looks aligned with the top of the
%% main column.
% \addnextpagesidebar[-1ex]{page3sidebar}

\end{document}
